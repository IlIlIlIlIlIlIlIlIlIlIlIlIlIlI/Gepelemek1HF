\documentclass[10pt, a4paper]{article} 
\usepackage[
a4paper,
left=2cm,      
right=2cm,     
top=3cm,     
bottom=3cm,    
headheight=4em 
]{geometry}
\usepackage[utf8]{inputenc}
\usepackage{amsmath,amsthm,amssymb,xfrac}
\usepackage{fancyhdr}
\usepackage[hungarian]{babel}
\usepackage{graphicx}
\usepackage{float}
\usepackage{comment}
\usepackage{siunitx}
\usepackage{hyperref}
\usepackage{siunitx}
\usepackage{natbib}
\usepackage{empheq}
\usepackage{wrapfig}
\usepackage{chngcntr}
\usepackage{physics}
\usepackage{mathtools}
\counterwithin{figure}{section}
\usepackage{titlesec}
\usepackage{dsfont}
\usepackage{pdfpages}
\usepackage{t1enc}
\usepackage{tabto}
\graphicspath{ {./images/} }
\setlength{\marginparwidth}{0pt}
\setlength{\marginparsep}{0pt}

\pagestyle{fancy}
\fancyhf{}
\cfoot{\thepage. oldal}
\lhead{
	\textbf{Gépelemek 1. Házi feladat}
	\\Kindlik Dániel
	\\AHU27Z
}
\newcounter{feladatcounter}
\newcommand{\feladat}{
	\stepcounter{feladatcounter}
	\begin{trivlist}
		\item[\hskip \labelsep {\bfseries
			{\arabic{feladatcounter}. Feladat:}}]\end{trivlist}}
\setlength{\parskip}{0.22em} 

\title{\includegraphics[width=175pt]{ BMElogo.png }\\Gépelemek 1. Házi feladat}
\author{Kindlik Dániel - AHU27Z}
\date{2025.10.03.}
\renewcommand*\contentsname{Tartalomjegyzék}
\begin{document}
	\maketitle
	\tableofcontents
	\newpage
	\setcounter{page}{2}
	\section{Bevezetés}
	A feladat a megadott adatokkal egy csővéget vakkarimával lezáró csavarkötés tervezése és az elemek szilárdságilag
	ellenőrzése.
	\subsection{Adatok}
	$p_{\text{ü}} = 35 mm\;\;D_N = 32 mm$
	\feladat
	\renewcommand{\arraystretch}{1.4}
	\begin{table}[h]
		\begin{tabular}{l|c|c}
			\textbf{Név}                              & \textbf{Jelölés} & \textbf{Érték} \\ \hline
			Karima külső átmérője                     & $D$                & 1 mm           \\
			Kiugrás mérete                            & $f$                & 2 mm           \\
			Tömítő felület külső átmérője             & $d_4$             & 3 mm           \\
			Csavar lyukkör átmérője                   & $d_2$             & 4 mm           \\
			Falvastagság                              & $s$                & 5 mm           \\
			Csavarok száma                            & $N$                & 6 mm           \\
			Csavarok közép átmérője                   & $K$                & 7 mm           \\
			Csavarok alapja és tömítési sík távolsága & $b$                & 8 mm           \\
			Kúp alsó átmérője                         & $d_3$             & 9 mm           \\
			Csavarok mérete                           & $M$                & 10 mm          \\
			Cső csatlakozás külső mérete              & $d_1$             & 11 mm          \\
			Karima magassága                          & $h$                & 12 mm         
		\end{tabular}
	\end{table}
	\renewcommand{\arraystretch}{1}
\end{document}